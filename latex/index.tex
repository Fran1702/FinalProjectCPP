Esta es la documentación del proyecto final de la catedra de Programación Orientada a Objetos en C++.

\subsection*{Objetivos}



El objetivo es simular la comunicación de 2 arduinos y una raspberry. Ademas, en la raspberry hay un thread corriendo que representa a una camara y envia datos de posición a la raspberry.

\subsection*{Esquema del programa}

Para esto se utilizan la siguiente estructura de disparo de threads\+:
\begin{DoxyItemize}
\item Main
\begin{DoxyItemize}
\item Arduino R\+: Corre el programa princiapl del Arduino R. En el se toman datos de una I\+MU y se envian a la raspberry. Ademas se leen mensajes que provienen de la raspberry para setear actuadores.
\item Arduino L\+: Corre el programa principal del Arduino L. En el mismo se toman datos de un sensor de distancia y se envian a la raspberry
\item Raspberry Pi
\begin{DoxyItemize}
\item Main\+: Corre el programa principal de la raspi, en el mismo se leen datos que provienen de los arduinos y se envian datos para setear los actuadores.
\item Camara\+: Lee datos de posición de la camara y los envia al programa principal de la raspi
\end{DoxyItemize}
\end{DoxyItemize}
\end{DoxyItemize}

Para la generación de los datos de la I\+MU y la camara se leen archivos .txt con los datos de aceleración y posición. En el caso del sensor de distancia, se generan numeros aleatorios para representar la posición medida.

\subsection*{Comunicaciones}

Para simular la comunicación utilizo queue con strings como elementos donde voy metiendo los mensajes que quiero mandar a otros dispositivos, y el otro dispositivo lee si le corresponde y si le corresponde lo pop, sino no hace nada.

Las siguientes queue se utilizan\+:
\begin{DoxyItemize}
\item queue 1\+: Comunicación entre los arduinos y raspi
\item queue 2\+: Comunicación entre la raspi y la camara
\end{DoxyItemize}

\subsection*{Envio de datos}

A continuación se muestra la estructura de los mensajes de cada dispositivo. \subsubsection*{I\+MU}

!\+D\+AT\+:I\+M\+U\+:xxxx\+:yyyy\+:zzzz\+:\#

\subsubsection*{C\+A\+M\+E\+RA}

!\+D\+AT\+:C\+A\+M\+:xxxx\+:yyyy\+:zzzz\+:\#

\subsubsection*{L\+A\+S\+ER}

!\+D\+AT\+:L\+A\+S\+:xxxx\+:\# 