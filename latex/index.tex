Esta es la documentación del proyecto final de la catedra de Programación Orientada a Objetos en C++.

\subsection*{Objetivos}



El objetivo es simular la comunicación de 2 arduinos y una raspberry. Para esto se utilizan la siguiente estructura de disparo de threads\+:
\begin{DoxyItemize}
\item Main
\begin{DoxyItemize}
\item Arduino R\+: Main
\item Arduino L\+: Main
\item Raspberry Pi
\begin{DoxyItemize}
\item Main
\item Camara
\end{DoxyItemize}
\end{DoxyItemize}
\end{DoxyItemize}

\subsection*{Comunicaciones}

Para simular la comunicación utilizo una Q\+U\+E\+UE con strings como elementos donde voy metiendo los mensajes que quiero mandar a otros dispositivos, y el otro dispositivo lee si le corresponde y si le corresponde lo pop, sino no hace nada.

Una queue es para la simulación entre los arduinos y la raspberry y otra queue par ala comunicación dentro de la raspberry con la camara.

\subsection*{Envio de datos}

A continuación se muestra la estructura de los mensajes de cada dispositivo. \subsubsection*{I\+MU}

!\+D\+AT\+:I\+M\+U\+:xxxx\+:yyyy\+:zzzz\+:\#

\subsubsection*{C\+A\+M\+E\+RA}

!\+D\+AT\+:C\+A\+M\+:xxxx\+:yyyy\+:zzzz\+:\#

\subsubsection*{L\+A\+S\+ER}

!\+D\+AT\+:L\+A\+S\+:xxxx\+:\# 