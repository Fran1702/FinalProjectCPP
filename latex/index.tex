\hypertarget{index_autotoc_md0}{}\section{Overview}\label{index_autotoc_md0}
Esta es la documentación del proyecto final de la catedra de Programación Orientada a Objetos en C++.\hypertarget{index_autotoc_md1}{}\section{Objetivos}\label{index_autotoc_md1}


El objetivo es simular la comunicación de 2 arduinos y una raspberry. Ademas, en la raspberry hay un thread corriendo que representa a una camara y envia datos de posición a la raspberry.\hypertarget{index_autotoc_md2}{}\section{Caracteristicas de los SE}\label{index_autotoc_md2}
\hypertarget{index_autotoc_md3}{}\subsection{Raspberry}\label{index_autotoc_md3}

\begin{DoxyItemize}
\item Intercambia mensajes con los Arduinos\+: Recibe mediciones de I\+MU y sensor de distancia y envia seteo de la bomba de agua.
\item Recibe datos del thread de la camara
\end{DoxyItemize}\hypertarget{index_autotoc_md4}{}\subsection{Arduino R}\label{index_autotoc_md4}

\begin{DoxyItemize}
\item Posee una I\+MU que mide aceleraciones en 3 ejes
\item Tiene conectada una bomba de agua a la cual se le setea la tensión
\end{DoxyItemize}\hypertarget{index_autotoc_md5}{}\subsection{Arduino L}\label{index_autotoc_md5}

\begin{DoxyItemize}
\item Tiene un sensor de posición laser
\end{DoxyItemize}\hypertarget{index_autotoc_md6}{}\section{Esquema del programa}\label{index_autotoc_md6}
El programa principal es el archivo \hyperlink{main_8cpp}{main.\+cpp}, donde se disparan los siguientes threads\+:
\begin{DoxyItemize}
\item AR\+: Corre el programa princiapl del Arduino R. En el se toman datos de una I\+MU y se envian a la raspberry. Ademas se leen mensajes que provienen de la raspberry para setear actuadores.
\item AL\+: Corre el programa principal del Arduino L. En el mismo se toman datos de un sensor de distancia y se envian a la raspberry
\item R\+PI\+: Corre el thread de la raspberry pi, dentro del cual se corren lo siguientes threads\+:
\begin{DoxyItemize}
\item Main\+: Corre el programa principal de la raspi, en el mismo se leen datos que provienen de los arduinos y se envian datos para setear los actuadores.
\item Camara\+: Lee datos de posición de la camara y los envia al programa principal de la raspi
\end{DoxyItemize}
\item Quit\+: Revisa si el usuario quiere terminar la simulacion (enter) en caso de querer terminar cierra todos los threads.
\end{DoxyItemize}

Para la generación de los datos de la I\+MU y la camara se leen archivos .txt con los datos de aceleración y posición. En el caso del sensor de distancia, se generan numeros aleatorios para representar la posición medida.\hypertarget{index_autotoc_md7}{}\section{Comunicaciones}\label{index_autotoc_md7}
Para simular la comunicación utilizo queue con strings como elementos donde voy metiendo los mensajes que quiero mandar a otros dispositivos, y el otro dispositivo lee si le corresponde y si le corresponde lo pop, sino no hace nada.

Las siguientes queue se utilizan\+:
\begin{DoxyItemize}
\item queue 1\+: Comunicación entre los arduinos y raspi
\item queue 2\+: Comunicación entre la raspi y la camara
\end{DoxyItemize}\hypertarget{index_autotoc_md8}{}\subsection{Envio de datos}\label{index_autotoc_md8}
A continuación se muestra la estructura de los mensajes de cada dispositivo. \subsubsection*{I\+MU}

!\+D\+AT\+:I\+M\+U\+:xxxx\+:yyyy\+:zzzz\+:\#

\subsubsection*{C\+A\+M\+E\+RA}

!\+D\+AT\+:C\+A\+M\+:xxxx\+:yyyy\+:zzzz\+:\#

\subsubsection*{L\+A\+S\+ER}

!\+D\+AT\+:L\+A\+S\+:xxxx\+:\# 